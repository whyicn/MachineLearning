\chapter{Data Understanding}

\section{Collect Initial Data}

The initial phase of our project entails gaining a comprehensive understanding of the dataset. Which involves a series of activities, beginning with data collection and selection, and extending to the crucial task of achieving balance within the dataset. By undertaking these data understanding activities, we lay the foundation for the subsequent phases of our data mining project. Armed with a comprehensive understanding of the dataset, we are better equipped to proceed with subsequent steps such as data preprocessing, feature engineering, and ultimately, model development.

\subsection{Initial Data Collection Report}
% List the data sources acquired, the method used to acquire them and any problems encountered (and any resolutions achieved)

To ensure the integrity and quality of our dataset, we have exercised diligence in selecting a trusted source. Our chosen dataset is from the Kaggle platform, renowned for its comprehensive and reliable datasets pertaining to various domains. Specifically, we have exclusively utilized the "train.csv" file, which contains the essential information necessary for our analysis.

\section{Describe Data}

Below we report the "gross" and "surface" properties of the acquired data and report the results.

\subsection{Data Description Report}
% Describe the data that has been acquired including its format, its quantity

The dataset has the following characteristics: 
\begin{itemize}
    \item Multivariate;
    \item 1460 rows representing the number of records;
    \item 81 columns representing the number of attributes in the dataset;
    \item 118,260  total data.
\end{itemize}
Out of the 47 attributes, we initially have the following division of data-types: 

\begin{itemize}
    \item 05 Date time attributes;
    \item 35 numeric attributes.
\end{itemize}
And rest are nominal attributes.
As we mentioned earlier we have 81 attribute and each attribute holds either numerical or categorical information. The following table-2.1 provides a clear idea on all the information and their description.

\begin{longtable}{|p{3cm}|p{3cm}|p{6.6cm}|}
%\begin{tabular}{|p{3cm}|p{3cm}|p{6.6cm}|}
\hline
\textbf{Name} & \textbf{Data Type} & \textbf{Description}                                             \\ \hline
SalePrice   & int64 & The property's sale price in dollars. This is the target variable \\ \hline
MSSubClass  & int64     & The building class \\ \hline
MSZoning    & object    & The general zoning classification \\ \hline
LotFrontage & object    & Linear feet of street connected to property \\ \hline
LotArea     & int64     & Lot size in square feet \\ \hline
Street      & object    & Type of road access \\ \hline
Alley       & object    & Type of alley access \\ \hline
LotShape    & object    & General shape of property \\ \hline
LandContour & object    & Flatness of the property \\ \hline
Utilities   & object    & Type of utilities available \\ \hline
LotConfig   & object    & Lot configuration \\ \hline
LandSlope   & object    & Slope of property \\ \hline
Neighborhood & object   & Physical locations within Ames city limits \\ \hline
TotalBsmtSF & int64     & Total square feet of basement area \\ \hline
BldgType    & object    & Type of dwelling \\ \hline
HouseStyle  & object    & Style of dwelling \\ \hline
OverallQual & int64     & Overall material and finish quality \\ \hline
YearBuilt   & int64     & Original construction date \\ \hline
RoofMatl    & object    & Roof material \\ \hline
Bedroom     & int64     & Number of bedrooms above basement level \\ \hline
Fireplaces  & int64     & Number of fireplaces \\ \hline
Functional  & object    & Home functionality rating \\ \hline
GarageType  & object    & Garage location \\ \hline
GarageArea  & int64     & Size of garage in car capacity \\ \hline
PavedDrive  & object    & Paved driveway \\ \hline
MiscVal     & int64     & \$Value of miscellaneous feature \\ \hline
Fence       & object    & Fence quality \\ \hline
YrSold      & int64     & Year Sold \\ \hline
SaleType    & object    & Type of sale \\ \hline
SaleCondition & object  & Condition of sale \\ \hline
%\end{tabular}
\end{longtable}

\section{Explore Data}

\subsection{Data Exploration Report}
% Describe results of data exploration
% Include graphs and plots to indicate data characteristics

To further explore the data, we conducted the following analyses:
\begin{itemize}
    \item \textbf{Descriptive Statistics:} We computed summary statistics for the numerical variables in the dataset, including measures such as mean, median, standard deviation, minimum, and maximum. This helped us gain insights into the central tendency, spread, and range of each variable.
    \item \textbf{Correlation Analysis:} We examined the correlation between the numerical variables and the target variable, "SalePrice." By calculating the correlation coefficients, we identified variables that had a strong positive or negative correlation with the target variable. This analysis helped us understand which features are most influential in determining house sale prices. In our case, some feature for example overall Quality, garage area, basement area has relatively high effect on target category.
    \item \textbf{Feature Distribution:} We visualized the distribution of numerical variables using histograms and box plots. This allowed us to identify potential outliers and skewed distributions that may require further investigation and data preprocessing. 
    \item \textbf{Categorical Variables:} We explored the categorical variables by examining the frequency distribution of each category. This helped us understand the distribution of different categorical features and their potential impact on house sale prices. We found that, prices rises as the garage area gets larger, also recently built houses has more value and so on. 
\end{itemize}

\section{Verify Data Quality}

\subsection{Data Quality Report}
% List the results of the data quality verification
% If there are any problems, suggest possible soluctions

Analyzing the occurrence of missing values we see that when the value is missing we find a “NaN” or “Na” values. We found values “NA” for columns “LotFrontage”, ”MasVnrArea”, ”GarageYrBlt”. But this attributes carry meaning for columns. Hence, we replaced the “NA” with ‘-1’ or the ‘Mean’ value of the column.  
With respect to the target class label i.e“SalePrice” attribute, we tried to check whether the dataset is balanced or not. We categorized the values as LOW, MEDIUM and HIGH. From the initial analysis of the data set the target class label was distributed as in the below three different values:
\begin{itemize}
    \item MEDIUM: 726 records;
    \item LOW: 619 records;
    \item HIGH: 115 records.
\end{itemize}
But despite the importance of balance, in this case, the distribution is not perfectly balanced, as the counts of each class vary.